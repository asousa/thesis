Surrounding the Earth are two shells of high-energy electrons and ions known as the Van Allen Radiation Belts, the dynamics of which are governed by a complex balance of sources and loss processes collectively resulting in populations which persist between days and months. The global distribution of lightning discharges has been theorized to be an important loss function for radiation belt electrons through a process known as Lightning- Induced Electron Precipitation (LEP). However, a lack of in situ measurements has precluded a quantitative global study. 

We present an investigation of the spatial and temporal morphology of lightning-induced electron precipitation. We seek to quantitatively determine the regions in space, time, and electron energy, where LEP may be of importance, using a combination of numerical raytracing and resonant interaction modeling, driven by a realistic measurement of global lightning activity from the GLD360 lightning detection network.

First, using cold plasma numerical raytracing, we provide an estimate of the persistent radio energy in the near-Earth space environment due to whistler-mode, VLF waves generated by terrestrial cloud-to-ground lightning flashes. Variation is explored with respect to L-shell, longitude, frequency, and geomagnetic activity using the Kp index. We include updated ionospheric absorption curves and an improved 4-dimensional interpolation algorithm. We find that the persistent energy decays logarithmically with increasing L-shell and frequency.

We then apply our raytracing and interpolation apparatus to simulate the precipitating electron flux resulting from single lightning strokes, and extrapolate over the GLD360 dataset to determine estimated locations of elevated electron precipitation. We compute the estimated lifetime of geomagnetically trapped electron populations, subject solely to LEP-induced losses, as a function of L-shell and electron energy. Our estimates are consistent with measurements in the 100keV -- 1 MeV energy range; however our results suggest that LEP may be a substantial loss mechanism for lower energy, suprathermal electrons in the 100 eV -- 10 keV range. We do not find any notable enhancement in the slot region, suggesting that resonant interactions with lightning-generated whistlers are not a significant contributor to slot region morphology.

Finally we present a design for a VLF radio receiver, intended for deployment on a 3-unit CubeSat bus, which can be used in conjunction with an onboard electron energy spectrometer to make direct, in situ measurements of precipitating electrons and their causative whistler waves. The receiver is designed with radiation tolerance in mind, and includes a novel onboard data processing system, implemented entirely in an field-programmable gate array (FPGA) using the Verilog hardware description language.