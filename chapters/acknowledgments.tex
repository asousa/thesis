This dissertation represents the culmination of my nearly seven years at Stanford, and I've been thinking about writing this section for the last three. And yet, sitting here at my desk at the University of Colorado, I find myself at a lack for words.

Going into a Ph.D program, you tend to think of a dissertation as a final, be-all, end-all answer, a pinnacle of unrelenting, focused, thoughtful research, culminating in The Great Answer to your questions. Now, at the end of my program, however, this dissertation feels oddly anticlimactic. My path to this point was by no means direct: graduate research is by nature a meander, a random walk trajectory through courses and projects and countless research papers. There's no straight line to the finish - but there is, with enough guidance, a drift. 

By no means would I say I've arrived at the fabled ``answer''. I don't feel a sense of enlightenment. My work here has left me with an ever-expanding set of questions, model improvements, approximations and assumptions in need of scrutiny. But in a sense that is the destination for the countless hopeful academics who have embarked on a Ph.D. At the end, there are only more questions.

My particular path through Stanford was not direct at all. I entered as a first year student in the VLF group, which focused on low-frequency radio waves in the magnetosphere. The VLF group was built on over 50 years of geophysics and radioscience research at Stanford, and consisted of a thriving collection of a half-dozen research staff and nearly twenty graduate students, under the direction of Dr. Umran Inan. I knew early on that the VLF group had reached the end of its lifespan going into it, but had been assured that things would continue smoothly, that there would be people to work with, and that I would be one of a handful of final students to carry the VLF banner.

Within my first couple years, I worked on several interesting and unique research opportunities -- beginning first with hardware design for a CubeSat (see chapter \ref{chapter:vpm}),

VLF waves traverse great distances, and research on these waves follows accordingly. The VLF group prided itself on maintaining a global network of student-maintained radio receivers, from Alaska to Antarctica. Less than two years in as a student, I found myself, along with fellow student Chris Young, dragging four heavy Pelican cases through the rush-hour crush of a train station in Kagoshima, Japan, as part of a ten-day stint installing three such receivers in Sapporo, Kagoshima, and Mt. Fuji. Similarly, I had been entrusted, along with a handful of other students, with designing a CubeSat -- which would go to space! -- which at the time felt both incredible and overwhelming. The excitement within the group was palpable, fueled by a driven professor and supported by the well-trained machinery of the senior research staff.

Unfortunately, things were somewhat built to spill, and as the quarters progressed steadily on, our ranks began to dwindle. Other groups brought in new students and new projects; but I was the last one in the door. The excitement dwindled; the research meetings grew smaller and smaller. Labs were shut down, equipment surplussed, and the open doors of professors remained frequently shut. 

As the group continued to shrink, my research advisor, Dr. Sigrid Close, graciously took me into her lab in the Aerospace Engineering department, along with VLF research scientists Dr. Robert Marshall and Dr. Ivan Linscott. Switching to the AA department was a fantastic opportunity to escape the loneliness and gloom of the Packard basement lab. Courses and research continued on, now in an active research group (and with daylight windows!). Not soon thereafter, however, Dr. Robert Marshall secured a coveted faculty position at CU Boulder. I found myself working on a project unrelated to my fellow students, and again lacked an in person mentor.

The fourth year was the hardest. By four years in, you've likely taken all the classes you need to, and hopefully have a start on a dissertation topic. But by four years, you're too far in to quit, and yet the end is nowhere in sight. Furthermore, with no remaining classes to count, you're truly cut free -- and without steady guidance, there's no obvious way forward.

-----------

There are countless people who, without their invaluable guidance, I would not be writing this section today. First and foremost, to my mentor, Dr. Robert Marshall, who has been perhaps the one constant thread throughout my seven years, first as a faculty mentor on the VPM project, then as a co-advisor in the AA department, and now as my PI at the University of Colorado. Bob, you're an inspiration.

To my advisor, Dr. Sigrid Close, who took me in despite having an already overwhelming cadre of students. I've always appreciated our conversations, and that you'd let them meander to topics across the board. As much as Bob would hone ideas in, Dr. Close would let brainstorming sessions run rampant.

To Dr. Antony Fraser-Smith, who as graciously served as my third committee member, especially while recovering from a heart surgery. Tony has been a part of the radioscience group at Stanford since the 1970s, and is a wellspring of interesting stories, entertaining anecdotes, and exudes the good-natured, jovial personality that makes geoscience a compelling world to work in. Furthermore, Dr. Fraser-Smith was responsible for putting together the Villard Fellowship for Students in Radioscience, in memory of the late Dr. Oswald Villard, which supported my first year of research, and was the deciding factor in my attending Stanford. 

To the researchers and engineers who kept the VLF group running: Dr. Ivan Linscott, Dr. Dave Lauben, Dr. Morris Cohen, who made themselves available for any and every question I had, no matter how trivial. A special thank you to Dr. Maria Spasojevic, who took the time to personally call me when I was applying to grad school.
 
To Jeff Chang, who singlehandedly ran every technical aspect of the VLF group, from the compute cluster to PCB layout to metal manufacturing.
 
To Professor Emiko Yasumoto in the Japanese language department, who showed special concern for my well-being during a particularly rough time. \begin{CJK}{UTF8}{min}??????????\end{CJK} \\

To my friends, colleagues, and lunch buddies from the VLF and SESS groups: Chris Young, with whom I've shared adventures across Japan; David Freese, for dragging me 5000 feet up Half Dome in the middle of the night; Patrick Blaes, who taught me everything I know about everything (no joke). To Rasoul Kabirzadeh, who started the same quarter as I did and went through many of the same struggles I encountered. To Nicholas, Sid, Anna, Lorenzo, and everyone in the SESS group who welcomed me in halfway through my program: Thank you all.

I owe a truly incalculable debt of gratitude to my community outside of campus, who without their constant companionship I most surely would have turned tail and returned to Seattle in 2012. To Ted Phares and Tristan Hudson, for 2am snacks and 2pm breakfast. To Damien Heiser and Wayne Tsai, for bringing me to Burning Man --Twice!\footnote{so far}. To Galen, Abe, JDT, Mike, Phil, Dan, Ricky, Hukka, Max, and everyone else: more than the classes, more than the research: you guys made me the person I am today, and I would not be here without you. Thank you.

To my former bosses and colleagues in the music industry, from my life before college, Bob Lang and Stuart Hallerman, for inspiring me to study electronics in the first place.

To my partner of nearly 9 years, Ty Logan: Thanks for putting up with me.

Finally, to my high school history teacher, Mark Rippy: In my senior year, I had to write a history paper and I refused to do it. Mark gave me some brief, offhand advice that hit me at just the right moment: ``Nobody else will do it for you." It was a bombshell, and it changed my life. It sounds silly, but those words became a mantra that got me through high school, an associate's degree, five years in the music industry, two bachelors degrees, a Master's degree, and now, a Stanford doctorate. Thank you.

