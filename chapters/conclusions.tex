Over the preceding chapters we have examined the global impact of lightning-generated whistlers, and associated electron precipitation, on the ionosphere and radiation belts. 

In chapter \ref{chapter:power} we modeled the persistent radio energy within the plasmasphere and magnetosphere due to lightning-generated whistlers; the energy is generally confined by the plasmapause, inside which there is little variation with respect to geomagnetic conditions. We find that the persistent energy decays logarithmically with increasing L-shell, as in figure \ref{fig:energy_density_vs_L_trendline}. We find a similar logarithmic trend in the radio frequency spectrum, in which the average bandwidth of lightning-generated energy decreases with increasing L-shell, as in figure \ref{fig:energy_density_vs_L_vs_freq}.

Chapter \ref{chapter:global_estimates} built on the ray tracing and interpolation methodology of chapter \ref{chapter:power} to examine the global impact of LEP, both as a magnetosphere-ionosphere energy coupling mechanism, and as a radiation belt depletion mechanism. We consider energy deposition from maximally and minimally populated radiation belts. We capture the seasonally-varying morphology of \cite{Gemelos2009}, and are in reasonable agreement with \cite{Meredith2007} for electron average electron lifetimes in the MeV energy range. Our simulation finds that while LEP is not a significant loss mechanism for highly-relativistic electrons in either the inner or outer belts, we find LEP to be a highly significant loss mechanism for suprathermal (100 eV - 10 keV) electrons in the inner radiation belt. We do not find any notable enhancement in slot-region losses, suggesting that LEP is not significant in the depletion of the slot region. However, we consider only the first 20 seconds of a propagating whistler, after which significant energy in the middle frequencies and middle L-shells remain; as was theorized by \cite{Bortnik2005}, this energy likely evolves into incoherent plasmaspheric hiss, which has been surmised elsewhere to be principal driver of the slot region.

Chapter \ref{chapter:VPM} detailed the practical issues of a CubeSat-based VLF receiver, designed for \emph{in-situ} measurements of LEP. While the receiver was constrained by the power and volume requirements of a CubeSat, this system included additional constraints not normally addressed in microscale / university-level satellite design. These constraints include designing for radiation-tolerant upgrades in mind, which, while of small significance for short-duration, low-earth-orbit missions, may be of greater significance for highly-elliptical orbits or future CubeSat missions to planets and objects outside the magnetosphere. Additionally, the receiver was designed as a peripheral device, with no volatile software onboard, using the hardware description language VERILOG.

\section{Suggestions for Future Work}

\subsection{Model Improvements}
\paragraph{Ray Tracing}
The morphology of LEP is driven primarily by the paths of incident whistlers; within our modeling, these paths are defined through raytracing in a modeled plasmasphere environment. Numerous improvements could be made here, with possible new science of interest, as the detail and realism in plasmasphere modeling is increased. Driven by both for a desire to generalize, and a lack of computational resources, we artificially constrain our raytracing to the meridional plane (as defined by the Earth's magnetic field). Furthermore we had difficulty in balancing accuracy versus reasonable timestepping (reducing error tolerances would frequently reduce our timestep to picoseconds), which resulted in erroneous deviation along the longitudinal axis. Exploring alternatives to the Runge-Kutta ODE solver, as well as incorporating the warm plasma corrections of 
\cite{Maxworth2017} may serve to reduce these errors.
Generally, at noon and midnight local time, we expect minimal gradients along the longitudinal axis, and the meridional-plane constraint should hold. However, lightning generally occurs around dusk and dawn (as in figure \ref{fig:Io_and_MLT}), where there may likely be stronger longitudinal gradients, which could impart a significant spreading or offset to the precipitation stencils of chapter \ref{chapter:global_estimates}. % different solvers, etc, etc, warm plasma effects, etc etc.

\paragraph{Interpolation and Granularity}
The energy and precipitation stencils could each benefit from a finer granularity in guide rays, and from finer frequency steps in interpolation. An additional, reasonably-straightforward modification to our interpolation scheme would be to consider the \emph{concave hull} mentioned in figure \ref{fig:convex_hulls}. Our interpolation algorithm at present assumes a convex hull, which will always span a volume equal or greater to the concave hull.

\paragraph{Backscatter and Altitude Effects}
Our precipitation codes do not include the Atmospheric Backscatter model of \cite{Cotts2011}; \citeauthor{Cotts2011} finds that a significant fraction of electrons bounce back after penetrating below 100km altitude (below which our model considers a particle to be lost). The rebounded particles may then interact with additional VLF wave activity, and possibly precipitate into the opposite hemisphere.

Of interest for transmitter-based measurements of LEP would be a more-thorough treatment of precipitating depth. As noted in \cite{Cotts2011}, there is a significant difference in precipitation likelihood between `grazing' precipitation and `deep' precipitation. Furthermore, the lower-energy, suprathermal electrons which comprise the bulk of our precipitation energy estimates likely precipitate at higher altitudes; VLF transmitter remote sensing techniques are sensitive dominantly to the lower edge of the ionosphere ($\sim$ 85 km), and may not be impacted by particles deposited closer to our 100 km threshold.
% raytracer improvements
% Incorporation of Atmospheric Backscatter codes
% Consideration of absorption altitude (low-energy stuff gets driven down *hard*, but always precipitates closer to 100km -- insignificant for transmitter studies?

\subsection{Global Trend Improvements}
\paragraph{Correlation between Filling Conditions and K$_p$}
% Filling conditions -- higher kp = higher filling, but lower wave activity?
The dynamics the radiation belt are marked by short-duration, sporadic filling events. These filling events may coincide with increased solar activity (possibly with a delay), which in turn may result in a higher K$_p$. Within our work, we consider precipitation from maximally and minimally filled radiation belts. However, we find less precipitation at higher K$_p$ values, due to the constricting nature of the plasmapause. Incorporating measured / historical filling data along with GLD360 and K$_p$ data may result in a better estimate of energy deposition, rather than an upper or lower bound.

\paragraph{Coherent vs Incoherent Methods}
In chapter \ref{chapter:global_estimates}, we determine global electron precipitation by taking a scaled sum of individual, (pseudo) coherent precipitation events using GLD360. However, numerous work exists on incoherent diffusion methods \citep{Kennel1966, Lyons1973, Glauert2005}. Rather than summing the precipitation after interaction, one could use the power estimates from chapter \ref{chapter:power}, and feed the determined distributions into an incoherent diffusion code. Such an implementation would be an excellent comparison between the coherent and incoherent diffusion approaches, which generally treat different physical systems -- coherent for LEP, and incoherent for chorus and hiss.
% Comparison to PADIE codes, etc

\subsection{Related Phenomena}
\paragraph{Cloud-to-Cloud lightning}
Our study is constrained to lightning discharges measured by the GLD360 dataset, which is designed to detect Cloud-to-Ground only. However, Cloud-to-Cloud lightning may be a significant, additional source of whistler waves. The single-flash modeling used in computing the energy and LEP stencils could be easily adapted to model a cloud to cloud discharge; one would need only to provide a new model of illumination (see section \ref{section:input_power}). 
% cloud-to-cloud
\paragraph{Transmitter-induced Precipitation}
Similarly, the stencil computation method could be adapted to study precipitation resulting from VLF transmitters. In contrast to the impulsive, broadband whistlers generated by lightning, VLF transmitters provide a constant, narrowband signal. It has been theorized that the constant effect of numerous global transmitters constitutes an anthropogenic driver of the so-called ``impenetrable barrier'' for relativistic electrons \citep{Baker2014, VLF_BUBBLE}.
% power and precipitation estimates for VLF transmitters

% different lightning datasets -- detection efficiency?

% Comparison to real data!

